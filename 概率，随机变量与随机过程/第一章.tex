%!mode::"TeX:UTF-8"
\documentclass{ctexart}
\usepackage{amsmath}
\usepackage{amssymb}
\usepackage{xltxtra}
\usepackage{mflogo,texnames}
\usepackage{graphicx}
\usepackage{listings}
\usepackage{listing}
\usepackage{xcolor}
\usepackage{xcolor}
\usepackage{enumerate}
\usepackage[colorlinks,linkcolor=blue]{hyperref}
\usepackage{cite}
\usepackage{color}
\usepackage{caption}
\usepackage{subfigure}
\usepackage[top=2.54cm,bottom=2.54cm,left=3.18cm,right=3.18cm]{geometry}
\author{右武卫大将军}
\title{概率,随机变量和随机过程(第一章)}
\begin{document}
    \maketitle
    \section{引言}
        概率论用来研究相继发生或同时发生的大量现象的平均特性。人们观测到,在许多领域中,当观测次数增加时,某些量的平均会趋向于一个常数;即使平均是对试验前特定的任何子序列进行,其值扔保持不变。概率论的母的就是用事件的概率来描述和预测这些平均值。事件$A$的概率是赋予这一事件的一个数$P(A)$,它可以解释为:
        
            如果实验重复进行$n$次,事件$A$发生$n_A$次,则当$n$足够大时,A发生的相对概率$n_A/n$以高度的确定性接近$P(A)$:
            \begin{equation}
                P(A) \approx n_A/n
                \label{f1-1}
            \end{equation}
            
        这种解释是不精确的。术语“以高度的把握性”,“接近”,“足够大”的含义都不明确。但是这种不精确是无法避免的,概率论只能以不准确的形式和物理现象相联系。
    \section{定义}
        \subsection{公理化定义}
            概率论中有三个基础性的公理:
            \begin{enumerate}
                \item 任一事件$A$的概率$P(A)$是赋予此事件的一个非负实数
                        $$P(A) \le 0$$
                \item 必然事件的概率等于1
                        $$P(S)=1$$
                \item 如果两个事件$A$和$B$是互斥的,则:
                        $$P(A\cup B) = P(A) + P(B)$$
            \end{enumerate}
        \subsection{相对频率定义}
            相对频率方法是基于下述定义:一事件的概率$P(A)$是极限
                \begin{equation}
                    P(A) = \lim_{n\leftarrow \infty} \frac{n_A}{n}
                    \label{f1-6}
                \end{equation}
            式中$n_A$是A是发生次数,$n$是试验次数。
                
        \subsection{古典定义}
            对概率的古典定义为:一个事件的概率$P(A)$可以不经过实际试验而先验的确定:
                \begin{equation}
                    P(A) = \frac{N_A}{N}
                    \label{f1-7}
                \end{equation}
            其中$N$是可能结果的总数,$N_A$是事件A的结果数。这种粗糙的定义忽略了每一个结果发生的可能性是不同的。这种有意地忽略是由不充分推理原理引出的,即:当没有先验知识时,我们只能假定事件$A_i$具有等概率性。即使规定了各个结果是等可能的,但同样存在一些问题:
            \begin{enumerate}
                \item 所谓的“等可能”也就是“等概率”的意义,但这一定义本身就是解决概率的定义问题。
                \item 所谓的“等可能”适用的问题极为有限。
                \item 这种“等可能”本身也是基于大量生活经验总结的,也并不是所谓的基于逻辑。
                \item 当可能的结果无穷多的时候,这种定义就失效了。
            \end{enumerate}
    \section{概率与归纳}
    \section{因果性和随机性}
        本质上,所有的物理现象或者社会现象都是确定性的,因果的,但前提是能够观测到所有的相关因素,由于这是无法实现的条件,所以,试验结果最终对观测者来说就是随机的。对于这一争议的回答是:物理学家并不关心什么是真正的,只关心什么是能够观测到的。
        
\end{document}
